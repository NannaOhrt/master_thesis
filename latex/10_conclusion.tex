\section{Conclusion}
This thesis has introduced the fundamental 
aspect there effect swaption prices. 
During the analysis the SABR model was chosen
to price swaption. It is important to be aware
of which choices we made along the way. 
Because model selection effect swaption prices. 
\\\\
Chapter 4 and Chapter 5 where determine for 
the selection of a two factor model. 
Since market data on swaption, showed a need
for a model there could handle stochastic 
volatility. From Chapter 7 i became clear 
that the various parameters in the SABR model, 
namely $\alpha$, $\beta$, $\rho$ and $\nu$ 
have different affect one the implied volatility,
and hence swaption price. 
From the analysis, we obtain the the same pattern in 
the $\alpha$ and $\beta$, therefor we chose 
to fix beta. Further in Chapter 8 we saw that we
where able to construct the observed implied volatility
from the market data. 
\\\\
Finally Chapter 9 made us aware
of the risk related to the SABR model and hence which market 
movement we should look a for, if swaption used take part 
in asset allocation.  So from the analysis 
in this thesis, swaption is a missing link in asset 
allocation. Since swaption can contribute with 
a different performance than other more common used asset 
class as equites and bonds.
