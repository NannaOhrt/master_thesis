\section{Conclusion}
This thesis has introduced the fundamental 
drivers that affect swaption prices. 
During the analysis the SABR model was chosen
to price swaptions. It is important to be aware
of which choices we made along the way, 
because model selection affect swaption prices. 
\\\\
Chapter 4 and Chapter 5 was motivation for 
the selection of a two factor model. 
We found that market data on swaptions showed a need
for a model that could handle stochastic 
volatility. From Chapter 7 i became clear 
that the various parameters in the SABR model, 
namely $\alpha$, $\beta$, $\rho$ and $\nu$ 
have different effects on the implied volatility,
and hence swaption prices. 
From the analysis, we obtained the same pattern in 
the $\nu$ and $\beta$, therefor we chose 
to fix the beta. Further in Chapter 8 we saw that we
were able to construct the observed implied volatility
from the market data. 
\\\\
Finally Chapter 9 made us aware
of the risk related to the SABR model and hence which market 
movement we should look out for, if swaptions are used
in asset allocation.  So from the analysis 
in this thesis, swaptions are a missing link in asset 
allocation, since swaptions can contribute with 
a positive performance in market conditions where other more commonly used assets 
classes such as equites and bonds fail to do so.
