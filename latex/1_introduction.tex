\section{Introduction}

This thesis investigates swaption pricing using the SABR model
and analyzes swaptions as a missing link in asset allocation. 
Data from Citi Velocity will be used and the analysis is 
based on the paper "Managing Smile Risk" by Hagen (2002) 
\cite{Smile}.
\\\\
Chapter 2 motivates for swaptions 
in asset allocation and explores the performance 
of various assets during different economic situations, 
utilizing data from Yahoo Finance and Citi Velocity. 
Understanding the construction of the swaption
is crucial for it's role in asset allocation.
\\\\
Chapter 3 introduces the various elements that affect 
swaptions, including interest rates, bonds, interest 
rate swaps, options, and pricing tools. This culminates 
in the presentation and formulation of swaption pricing. 
The thesis then examines the Vasicek model in Chapter 4, 
formulating bond pricing using this model and discusses its 
weak spots.
\\\\
Chapter 5 addresses the assumption of constant volatility, 
used in both the Black-Scholes and Vasicek models, 
and its applicability to swaptions. Chapter 6 introduces 
swaption terminology and outlines swaption data.
\\\\
The analysis continues with introducing the SABR model, which is 
more suitable for swaption pricing. Chapter 7 covers the SABR model 
and investigates its parameters to enable swaption pricing, 
concluding with the formulation of swaption pricing using 
the SABR model.
\\\\
Finally, the thesis digs into the risk aspects associated 
with the SABR model in Chapter 9. This Chapter provides a 
comprehensive examination of the potential risks and 
challenges that might arise when using the SABR model 
for swaption pricing.
Chapter 10 concludes the investigation presented in the thesis. 
