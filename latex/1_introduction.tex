\section{Introduction}

This thesis investigates swaption pricing using the SABR model
and analyzes swaptions as a missing link in asset allocation. 
Data from Citi Velocity will be used, and the analysis is 
based on the paper "Managing Smile Risk" by Hagen (2002) 
\cite{Smile}.
\\\\
Chapter 2 motivates the use of swaptions as a missing link 
in asset allocation. This chapter explores the performance 
of various assets during different economic situations, 
utilizing data from Yahoo Finance and Citi Velocity. 
Understanding the construction of the swaption instrument 
is crucial for it's role in asset allocation.
\\\\
Chapter 3 introduces the various elements that affect 
swaptions, including interest rates, bonds, interest 
rate swaps, options, and pricing tools. This culminates 
in the presentation and formulation of swaption pricing. 
The thesis then examines the Vasicek model in Chapter 4, 
formulating bond pricing using this model and noting it's 
weak spots.
\\\\
Chapter 5 addresses the assumption of constant volatility, 
common in both the Black-Scholes and Vasicek models, 
and its applicability to swaptions. Chapter 6 introduces 
swaption terminology and outlines swaption data.
\\\\
The analysis continues with the SABR model, which better 
fits swaption pricing. Chapter 7 covers the SABR model 
and investigates its parameters to enable swaption pricing, 
concluding with the formulation of swaption pricing using 
the SABR model.
\\\\
Finally, the thesis delves into the risk aspects associated 
with the SABR model in Chapter 9. This chapter provides a 
comprehensive examination of the potential risks and 
challenges that might arise when using the SABR model 
for swaption pricing. It explores various scenarios and 
conditions under which the model might fail or produce 
unreliable results, offering insights into how these 
risks can be managed or mitigated.
Chapter 10 concludes the investigation presented in the thesis. 
