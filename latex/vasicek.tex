\section{One-Factor Short-Rate Model}
The risk-free short rate, r, is sometimes referred to as the instantaneous short rate. 
The concept is used in finance modeling to represent the continuously compounded interest rate for 
short time intervals. The short rate, r, is often modeled using stochastic deferential equations in 
mathematical finance. Some typically models for modeling the short rate is the Vacisek model and the CIR model, 
later the Vacisek model will be covered. When pricing derivatives as bonds and options, the price depends on 
the process followed by r in the risk-neutral world. \cite{Hull}
\\\\
As discussed in the section risk Neutral Measure, $r_t$ can be looked at the locally risk-free 
rate from a continuously compounded bank account $B(t)= \exp \Big[\int_{0}^{t} r(s) ds \Big]$. 
Where the bank account has the dynamic listed in \autoref{bank1} and \autoref{bank2}.
Postulation the the considered market is arbitrage-free, with due to the First Fundamental Theorem of Asset Pricing, 
if stating that there exist a probability measure $\QQ$, equivalent to $\PP$, all asset prices discounted by $B(t)$
are $\QQ$-martingales. In other words under the considered market for ant T we have that 
\begin{align}
    \frac{P(0,T)}{B(0)} = P(0,T) = \EE^{\QQ} \Big( \frac{P(T,T)}{B(T)}\Big) = \EE^{\QQ} \Big( \frac{1}{B(T)}\Big) 
    = \EE^{\QQ} \Big( \exp \Big[ - \int_{0}^{t}r(s)ds \Big] \Big)
    \label{dist}
\end{align}
where $P(0,T)$ the price at time zero of the asset and know that $P(T,T)=1$. So \autoref{dist} says that 
the time-zero price of the asset are $\QQ$-expectations the payoff. \cite{Bermudan}

\subsection{Vacisek}