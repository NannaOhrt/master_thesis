\section{Mathematics of pricing swaptions}
\textcolor{red}{Look at Swaption pricing and isolating volatility exposure.}
\\\\
To determine swaptions prices, it is important to understand which things there affects the price of the swaption. 
This chapter simplifies these concepts by explaining interest rates, bonds, swaps, and options, 
and then shows how they come together to determine the price of a swaption.
\subsection{Time value of money}
Understanding the concept of interest rates begins with the fundamental idea that a dollar today holds 
more value than the same dollar in the future. To understand these concept, a discount factor is introduce 
\begin{align*}
    B(t,T) = \text{value at time t of a dollar received at time T}
\end{align*} 
$B(t,T)$ refer to a contract that pays one dollar maturity, T, which can be illustrated as below
\begin{align*}
    t & < T \rightarrow B(t,T) < 1 \\
    t & = T \rightarrow B(t,T) = 1
\end{align*}
The concept of the "time value of money" it asserts that the value of money today is worth more than
the same amount in the future due to its potential earning capacity, inflation, and risk.
This principle underpins various financial decisions, including investing, borrowing,
and pricing financial instruments. Essentially, it recognizes that a dollar received today can be invested 
and earn interest over time, thereby increasing its value. Conversely, a dollar received in the future
is subject to uncertainty and may not retain its purchasing power due to inflation or other factors.
The discount factor represents the present value of future cash flows, taking into account the time value of money.
It reflects the idea that receiving a certain amount of money in the future is less valuable than receiving 
the same amount today.
\\\\
The yield is defined as the singular constant interest rate, denoted as $r$,
which has an equivalent impact to the discount factor $B(t, T)$ when compounded continuously.
\begin{align*}
    B(t,T)= e^{r(T-t)}
\end{align*}

\subsection{The yield curve}
Where the concept "time value of money" and the discount factor are fundamental concepts used to assess the present value of future
cash flows, while the yield curve provides insights into market expectations regarding future interest rates.
Understanding the interplay between these concepts is crucial for making informed investment decisions and pricing
financial instruments.
\\\\
The yield curve is a graphical representation illustrating the interest rates (bond yields) for various maturities.
Yield curve can provide a intuition about future interest rates and give insight in the bond market today. 
The general intuition is that longer-term rates is higher then short-term rates, which in other words means that a
lager premium is expect for lending money over a longer period of time. This case sketches a yield cure with a 
positive slope. It is important to know that the yield curve for a given interest change over time. This is supported
by the yield curves illustrated below, where the same interest rate is displayed for the same maturities, but the 
data is from different days. 
\\\\
\textcolor{red}{make yield curve}

\subsection{Zero Coupon Bonds}

\subsection{Forward rates}

\subsection{Bonds}

\subsection{Financial derivatives}

\subsection{Interest rate swaps}

\subsection{Options}

\subsection{Swaptions}