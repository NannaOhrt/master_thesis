\section{Mathematics of pricing swaptions}
\textcolor{red}{Look at Swaption pricing and isolating volatility exposure.}
\\\\
To determine swaptions prices, it is important to understand which things there affects the price of the swaption. 
This chapter simplifies these concepts by explaining interest rates, bonds, swaps, and options, 
and then shows how they come together to determine the price of a swaption.
\subsection{Time value of money}
Understanding the concept of interest rates begins with the fundamental idea that a dollar today holds 
more value than the same dollar in the future. To understand these concept, a discount factor is introduce 
\begin{align*}
    B(t,T) = \text{value at time t of a dollar received at time T}
\end{align*} 
$B(t,T)$ refer to a contract that pays one dollar maturity, T, which can be illustrated as below
\begin{align*}
    t & < T \rightarrow B(t,T) < 1 \\
    t & = T \rightarrow B(t,T) = 1
\end{align*}
The concept "time value of money" it asserts that the value of money today is worth more than
the same amount in the future due to its potential earning capacity, inflation, and risk.
This principle underpins various financial decisions, including investing, borrowing,
and pricing financial instruments. Essentially, it recognizes that a dollar received today can be invested 
and earn interest over time, thereby increasing its value. Conversely, a dollar received in the future
is subject to uncertainty and may not retain its purchasing power due to inflation or other factors.
The discount factor represents the present value of future cash flows, taking into account the time value of money.
It reflects the idea that receiving a certain amount of money in the future is less valuable than receiving 
the same amount today.
\subsection{Zero coupon bonds}
One of the most common applications of the concept "time value of money" is zero coupon bonds. 
By there construction the mechanism of "time value of money" is present. This instrument 
have the common property that they provide the owner with a deterministic cash flow. 
\begin{definition}\label{def:zcb}
    A zero coupon bond with maturity data T, also called a T-bons, is a contract which 
    guarantees the holder one dollar to be paid on the date T. The price at time t of 
    a bond with maturity data T is denoted by p$(t,T)$ \cite{Bjork} 
\end{definition} 
\subsection{The yield curve}
Where the concept "time value of money" and the discount factor are fundamental concepts used to assess the present value of future
cash flows, while the yield curve provides insights into market expectations regarding future interest rates.
Understanding the interplay between these concepts is crucial for making informed investment decisions and pricing
financial instruments. The yield curve is a graphical representation illustrating the interest rates (bond yields) for various maturities.
Yield curve can provide a intuition about future interest rates and give insight in the bond market today. 
The general intuition is that longer-term rates is higher then short-term rates, which in other words means that a
lager premium is expect for lending money over a longer period of time. This case sketches a yield cure with a 
positive slope.
\subsection{Interest rates}
\subsubsection{Spot rates}
The spot rate represents the yield-to-maturity of a zero coupon bond,
while the forward rate refers to the anticipated interest rate in the 
future. The definition for determined spot rates is listed 
below
\begin{definition}\label{def:spot}
    The simple spot rate for $[S,T]$, henceforth referred to as the 
    LIBOR spot rate, is defined as \cite{Bjork} 
    \begin{align*}
        L(t;S,T) = - \frac{p(t,T)-p(t,S)}{(T-S)p(t,T)}
    \end{align*}
\end{definition} 
\subsubsection{Forward rates}
Forward rates play a crucial role in financial markets, particularly in the realm of interest rate analysis and 
derivative pricing. They represent the interest rate applicable to a future period, agreed upon today.
Understanding forward rates requires grasping the concept of forward contracts and the expectations theory of interest rates.
Forward rates can be derived from the yield curve. The yield curve plots the yields of bonds with different maturities.
By analyzing the yield curve, one can infer the implied forward rates for future periods. For example, 
the forward rate between year 1 and year 2 is the rate at which an investor can borrow or lend money for the period
between year 1 and year 2, starting at year 1.
\\\\
Lets consider three time points on the yield curve $t=0,1,2$, where it is assumed
that $t_0 < t_1 < t_2$. At time $t_0$ we have the spot rates $p(t_0,t_1)$ and $p(t_1,t_2)$,
which represent the yields for bonds maturing at time $t_1$ and $t_2$ respectively.
Hence the forward rate, $R(t_1,t_2)$, can med determined using the equation below
\begin{align*}
    R(t_1,t_2)= \frac{(1+p(t_0,t_2))^2}{(1+p(t_0,t_1))}-1
\end{align*}
Imagine investing one dollar in a one-year zero-coupon bond, $B(t_0,t_1)$,
and instantly reinvesting the money received at time $t_1$ in a new one-year zero-coupon bond,
$B(t_1,t_2)$, at rate $R(t_1,t_2)$. This strategy should yield the same return as investing 
one dollar in a two-year zero coupon bond $B(t_0,t_2)$ and holding it for two years. 
This strategy illustrated the idea of forward rates. Let us then look a the general
formula for forward rates. 
\begin{definition}\label{def:forward}
    The continuously compounded forward rate for $[S,T]$ contracted at t is defined
    as \cite{Bjork} 
    \begin{align*}
        R(t;S,T)= - \frac{\log p(t,T)- \log p(t,S)}{(T-S)} 
    \end{align*}
\end{definition} 

\subsection{Bonds}

\subsection{Financial derivatives}

\subsection{Interest rate swaps}

\subsubsection{xIBOR rates}

\subsection{Options}

\subsection{Swaptions}