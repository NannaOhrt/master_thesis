\section{Risk Related to the SABR model} \label{risk_mang}
In this Section risk related to the SARB model will be covered. 
This will be a adding to the investigation of the parameters in the SABR model.
We will start this section by  reminding yourself of the value for
a European call swaption according to the SABR model. 
Then the section will continuing covering the related risk, also called referred to as sensitives in 
the SABR model. 
\\\\
So du to the examination of the Black Scholes model we have that \cite{Smile}
\begin{align}
    V_{\text{call}} &= D(t_{\text{set}})fN(d_1) - KN(d_2) \label{V_call} 
\end{align}
with
\begin{equation}
    d_{1,2} = \frac{\log \frac{f}{K} \pm \frac{1}{2}\sigma_B^2 t_{\text{ex}}}{\sigma_B \sqrt{t_{\text{ex}}}}
\end{equation}
where $t_{\text{set}}$ is the settlement date and $t_{\text{ex}}$ is the exercise date.
According to the SABR model, the value of a call is 
\begin{align}
    V_{\text{call}}= BS(f, K, \sigma_{\text{B}}(K,f),t_{\text{ex}})
\end{align}
with  
\begin{align}
    \sigma_{\text{B}}(K,f) \equiv \sigma_{\text{B}}(K,f;\alpha, \beta, \rho, \nu)
\end{align}
where the implied volatility $\sigma_{\text{B}}(K,f) $ is given as below
\begin{equation}
    \sigma_B(K, f) = \frac{\alpha}{(fK)^{(1-\beta)/2}} \left\{ 1 + \frac{(1-\beta)^2}{24} \log^2 \frac{f}{K} + \frac{(1-\beta)^4}{1920} \log^4 \frac{f}{K} + \ldots \right\} \left( \frac{z}{x(z)} \right)
    \label{eg_2}
\end{equation}
where
\begin{align}
    z &= \frac{\nu}{\alpha}(fK)^{(1-\beta)/2} \log \frac{f}{K}, \\
\end{align}
and x(z) is defined by
\begin{align}
    x(z) &= \log \left\{ \frac{\sqrt{1-2\rho z + z^2} + z - \rho}{1 - \rho} \right\}
\end{align}
For the special case of ATM options, options strike at $K = f$, this formula reduces to
\begin{equation}
    \sigma_{ATM} = \sigma_B(f, f) = \frac{\alpha}{f^{1-\beta}} \left\{ 1 + \left( \frac{(1-\beta)^2}{24} \frac{\alpha^2}{f^{2-2\beta}} + \frac{\rho \beta \nu}{4} \frac{\alpha}{f^{1-\beta}} + \frac{2-3\rho^2}{24} \nu^2 \right) t_{\text{ex}} + \ldots \right\}
    \label{sigma_ff_risk}
\end{equation}
\\\\
Lets then remind yourself of the interpretation of the parameters in the SABR model. 
\begin{itemize}
    \item $F_0$ \text{---} Initial forward rate or asset price.
    \item $\alpha_0$ \text{---} Initial volatility.
    \item $\beta$ \text{---} Elasticity parameter.
    \item $\nu$ \text{---} Volatility of the volatility parameter.
    \item $\rho$ \text{---} Correlation between the asset price and its volatility.
\end{itemize}
\noindent
So new we are ready to look at the risk related to the SABR model.
We will cover some different types of risk, namely vega, vanna, volga and delta risk. 
The various types of risk, are related to the parameters in the SABR. 
From your analysis of SABR model we argued that it is common practice, 
to fixed beta. Hence we will not look into any risk related to the beta parameter. 
But all the other parameters risk will be study. 
\\\\
During the examination of the risk related to the SABR will will look at 
the derivatives with respect to the different parameters. 
First we will look at the vega risk, which is the risk related to the $\alpha$ parameter. 
Hence it is also the risk related to the changes in the  volatility.
\\\\
From differentiating \autoref{V_call} with respect to $\alpha$ we have that
\begin{align}
    \frac{\partial V_{\text{call}}}{\partial \alpha} = 
    \frac{\partial \text{BS}}{\partial \sigma_B} \cdot \frac{\partial 
    \sigma_B(K, f; \alpha, \beta, \nu)}{\partial \alpha}
\end{align}
When taking about the risk related to the change in volatility, $\alpha$, 
the risk changes by a unit amount. We also note the it is common used finance to 
scale vega, so it represents the change in value when the AMT volatility change by a unit amount \cite{Smile}.
Then we note that 
\begin{align}
    \delta  \cdot \sigma_{\text{ATM}} =
     \Big(\frac{\partial \sigma_{\text{ATM}}}{\partial \alpha} \Big) \cdot \delta \alpha \label{eg_1}
\end{align}
where $\delta$ represents the changes. Due to \autoref{eg_1} we can write the vega risk as below
\begin{align}
    \text{vega} \equiv  \frac{\partial V_{\text{call}}}{\partial \alpha} 
   = \frac{\frac{\partial \sigma_B(K, f; \alpha, \beta, \nu)}{\partial \alpha}}
    {\frac{\partial \sigma_{\text{ATM}}(f; \alpha, \beta, \nu)}{\partial \alpha}} \label{eg_3}
\end{align}
 where we have that $\sigma_{\text{ATM}}(f) = \sigma_{\text{B}} (f,f)$  is given as listed \autoref{sigma_ff_risk}
 above. 
Then we note that  $\frac{\partial \sigma_B}{\partial \alpha} \approx \frac{ \sigma_B}{ \alpha}$ 
and $\frac{\partial \sigma_{\text{ATM}}}{\partial \alpha} \approx \frac{ \sigma_{\text{ATM}}}{\alpha}$,
hence the vega risk can be expressed as 
\begin{align}
\text{vega} \approx \frac{\partial \text{BS}}{\partial \sigma_B} \cdot
 \frac{\sigma_B(K,f)}{\sigma_{\text{ATM}}(f)} \cdot \frac{\partial \text{BS}}{\partial \sigma_B} 
 \cdot \frac{\sigma_B(K,f)}{\sigma_B(f,f)}
\end{align}
So when we work wit the SABR model,
the Vega risks for a swaption at various strike prices are calculated by adjusting the implied volatility at each strike 
K. This adjustment is proportional to the existing implied volatility 
$\sigma_B(K,f)$
at that strike. In this way, using \autoref{eg_3}, we adjust the
volatility curve in a proportional manner, rather than shifting 
it uniformly, allowing us to more accurately calculate the total
Vega risk for a portfolio of options.
\\\\
The remind yourself of how $\rho$ and $\nu$ er determine in the SABR model.
We learned that $\rho$ and $\nu$ are determine by fitting the implied volatility
surface observed in the market. But there are also risk related to these 
to parameters. So now we will look into the vanna and volga risk. 
First we note that the vanna risk in the SABR model, is related 
to the  parameter $\rho$. Secondly the volga risk is related
to the $\nu$ risk. The lingo of volga comes from v volatility of gamma
\cite{Smile}.
\\\\
Then lets look at how the two risk is determined. 
\begin{align*}
    \text{vanna} &= \frac{\partial V_{\text{call}}}{\partial \rho} = \frac{\partial BS}{\partial \sigma_B} \frac{\partial \sigma_B(K, f; \alpha, \beta, \nu)}{\partial \rho} \\
    \text{volga} &= \frac{\partial V_{\text{call}}}{\partial \nu} = \frac{\partial BS}{\partial \sigma_B} \frac{\partial \sigma_B(K, f; \alpha, \beta, \nu)}{\partial \nu}
\end{align*}
Then we remind yourself of the interpretation of $\rho$,
which was the correlation between the asset price and the its 
volatility. Which make since since the vanna risk express
the risk ro the skew increase. In other word the vanna risk 
measures the sensitives of an option  vega to change in 
the underlying asset price. Next we will cover the volga risk.
The volga risk is related to the $\nu$ parameter in the SABR model. 
The volga risk expresses the risk to the smile shape and curvature,
which we also saw during the investigating of the SABR model. 
\\\\
Finally we will look at the delta risk, we will denoted
the delta risk $\Delta$. The delta risk is related to f, the 
forward rate or the price of the underlying asset. 
During the investigating of the SABR model, we made
some choice regrading the determining of the alpha parameter.
This choice also effect the delta risk. So we note that
we chose the approach of determine the alpha, from solve the
$\sigma_{\text{ATM}}$, where $\sigma_{\text{ATM}}$ is given as 
listed in \autoref{sigma_ff_risk} above. 
Then we differentiating with respect to f, to obtain the delta risk. 
\begin{align}
    \Delta = \frac{\partial \text{BS}}{\partial f} +
     \frac{\partial \text{BS}}{\partial \sigma_\beta} 
     \Big({\frac{\partial \sigma_\beta(K; f, \alpha, \beta, \rho, \nu)}
     {\partial f}  + \frac{\partial \sigma_\beta(K; f, \alpha, \beta,
     \rho, \nu)}{\partial \alpha} 
     \frac{\partial \sigma(\text{ATM}, f)}{\partial f}}\Big)
\end{align}
The first term represents the ordinary Delta risk calculated from 
the Black's model. The second term adds the SABR model's
 specific adjustment for the change in implied volatility, 
$\sigma_B$, caused by changes in the forward price f.
the last term represents the adjustment needed to maintain a constant 
$\sigma_{\text{ATM}}$ while f changes.
We also note that the last term will be zero, if $\beta=1$ \cite{Smile}.
\\\\
Now we have covered all the risk or sensitives related to the SABR model. 
The tends we saw during the investigating of the SABR model, reflects
the same patterns regrading the various risk related to the model. 