\section{Risk Related to The SABR Model} \label{risk_mang}
In this Chapter, risks related to the SARB model will be covered. 
This will be adding to the investigation of the parameters in the SABR model.
We will start this Chapter by reminding ourself of the value for
a European call swaption according to the SABR model. 
Then the Chapter will continuing covering the related risk, also called referred to as sensitives in 
the SABR model. 
\\\\
So due to the examination of the Black Scholes model we have that \cite{Smile}
\begin{align}
    V_{\text{call}} &= D(t_{\text{set}})fN(d_1) - KN(d_2) \label{V_call} 
\end{align}
with
\begin{equation}
    d_{1,2} = \frac{\log \frac{f}{K} \pm \frac{1}{2}\sigma_B^2 t_{\text{ex}}}{\sigma_B \sqrt{t_{\text{ex}}}}
\end{equation}
where $t_{\text{set}}$ is the settlement date and $t_{\text{ex}}$ is the exercise date.
According to the SABR model, the value of a call is 
\begin{align}
    V_{\text{call}}= BS(f, K, \sigma_{\text{B}}(K,f),t_{\text{ex}})
\end{align}
with  
\begin{align}
    \sigma_{\text{B}}(K,f) \equiv \sigma_{\text{B}}(K,f;\alpha, \beta, \rho, \nu)
\end{align}
where the implied volatility $\sigma_{\text{B}}(K,f) $ is given as below
\begin{equation}
    \sigma_B(K, f) = \frac{\alpha}{(fK)^{(1-\beta)/2}} \left\{ 1 + \frac{(1-\beta)^2}{24} \log^2 \frac{f}{K} + \frac{(1-\beta)^4}{1920} \log^4 \frac{f}{K} + \ldots \right\} \left( \frac{z}{x(z)} \right)
    \label{eg_2}
\end{equation}
where
\begin{align}
    z &= \frac{\nu}{\alpha}(fK)^{(1-\beta)/2} \log \frac{f}{K}, \\
\end{align}
and x(z) is defined by
\begin{align}
    x(z) &= \log \left\{ \frac{\sqrt{1-2\rho z + z^2} + z - \rho}{1 - \rho} \right\}
\end{align}
For the special case of ATM options, options strike at $K = f$, this formula reduces to
\begin{equation}
    \sigma_{ATM} = \sigma_B(f, f) = \frac{\alpha}{f^{1-\beta}} \left\{ 1 + \left( \frac{(1-\beta)^2}{24} \frac{\alpha^2}{f^{2-2\beta}} + \frac{\rho \beta \nu}{4} \frac{\alpha}{f^{1-\beta}} + \frac{2-3\rho^2}{24} \nu^2 \right) t_{\text{ex}} + \ldots \right\}
    \label{sigma_ff_risk}
\end{equation}
\\\\
Lets then remind ourself of the interpretation of the parameters in the SABR model. 
\begin{itemize}
    \item $F_0$ \text{---} Initial forward rate or asset price.
    \item $\alpha_0$ \text{---} Initial volatility.
    \item $\beta$ \text{---} Elasticity parameter.
    \item $\nu$ \text{---} Volatility of the volatility parameter.
    \item $\rho$ \text{---} Correlation between the asset price and its volatility.
\end{itemize}
\noindent
So now we are ready to look at the risk related to the SABR model.
We will cover some different types of risk, namely vega, vanna, volga and delta risk. 
The various types of risk, are related to the parameters in the SABR. 
From our analysis of the SABR model we argued that it is common practice, 
to fix the beta. Hence we will not look into any risk related to the beta parameter. 
But all the other parameters risk will be studied. 
\\\\
During the examination of the risk related to the SABR will will look at 
the derivatives with respect to the different parameters. 
First we will look at the vega risk, which is the risk related to the $\alpha$ parameter. 
Hence it is also the risk related to the changes in the  volatility.
\\\\
From differentiating \autoref{V_call} with respect to $\alpha$ we have that
\begin{align}
    \frac{\partial V_{\text{call}}}{\partial \alpha} = 
    \frac{\partial \text{BS}}{\partial \sigma_B} \cdot \frac{\partial 
    \sigma_B(K, f; \alpha, \beta, \nu)}{\partial \alpha}
\end{align}
When talking about the risk related to the change in volatility, $\alpha$, 
the risk changes by a unit amount. We also note the it is commonly used in finance to 
scale vega, so it represents the change in value when the AMT volatility change by a unit amount \cite{Smile}.
Then we note that 
\begin{align}
    \delta  \cdot \sigma_{\text{ATM}} =
     \Big(\frac{\partial \sigma_{\text{ATM}}}{\partial \alpha} \Big) \cdot \delta \alpha \label{eg_1}
\end{align}
where $\delta$ represents the changes. Due to \autoref{eg_1} we can write the vega risk as below
\begin{align}
    \text{vega} \equiv  \frac{\partial V_{\text{call}}}{\partial \alpha} 
   = \frac{\frac{\partial \sigma_B(K, f; \alpha, \beta, \nu)}{\partial \alpha}}
    {\frac{\partial \sigma_{\text{ATM}}(f; \alpha, \beta, \nu)}{\partial \alpha}} \label{eg_3}
\end{align}
 where we have that $\sigma_{\text{ATM}}(f) = \sigma_{\text{B}} (f,f)$  is given as listed in \autoref{sigma_ff_risk}
 above. 
Then we note that  $\frac{\partial \sigma_B}{\partial \alpha} \approx \frac{ \sigma_B}{ \alpha}$ 
and $\frac{\partial \sigma_{\text{ATM}}}{\partial \alpha} \approx \frac{ \sigma_{\text{ATM}}}{\alpha}$,
hence the vega risk can be expressed as 
\begin{align}
\text{vega} \approx \frac{\partial \text{BS}}{\partial \sigma_B} \cdot
 \frac{\sigma_B(K,f)}{\sigma_{\text{ATM}}(f)} \cdot \frac{\partial \text{BS}}{\partial \sigma_B} 
 \cdot \frac{\sigma_B(K,f)}{\sigma_B(f,f)}
\end{align}
So when we work with the SABR model,
the vega risks for a swaption at various strike prices are calculated by adjusting the implied volatility at each strike 
K. This adjustment is proportional to the existing implied volatility 
$\sigma_B(K,f)$
at that strike. In this way, using \autoref{eg_3}, we adjust the
volatility curve in a proportional manner, rather than shifting 
it uniformly, allowing us to more accurately calculate the total
vega risk for a portfolio of options.
\\\\
Then we remind ourself of how $\rho$ and $\nu$ are determined in the SABR model.
We learned that $\rho$ and $\nu$ are determined by fitting the implied volatility
surface observed in the market. But there are also risk related to these 
to parameters. So now we will look into the vanna and volga risk. 
First we note that the vanna risk in the SABR model, is related 
to the  parameter $\rho$. Secondly the volga risk is related
to the $\nu$ risk. The lingo of volga comes from volatility of gamma
\cite{Smile}. Gamma measure the convexity of a derivative's value in relation to the underlying asset. 
\\\\
Then lets look at how the two risk is determined. 
\begin{align*}
    \text{vanna} &= \frac{\partial V_{\text{call}}}{\partial \rho} = \frac{\partial BS}{\partial \sigma_B} \frac{\partial \sigma_B(K, f; \alpha, \beta, \nu)}{\partial \rho} \\
    \text{volga} &= \frac{\partial V_{\text{call}}}{\partial \nu} = \frac{\partial BS}{\partial \sigma_B} \frac{\partial \sigma_B(K, f; \alpha, \beta, \nu)}{\partial \nu}
\end{align*}
\\
Then we remind ourself of the interpretation of $\rho$,
which was the correlation between the asset price and the its 
volatility. Which make sense since the vanna risk express
the risk to the skew increase. In other words, the vanna risk 
measures the sensitivity of an options vega to change in 
the underlying asset price. Next we will cover the volga risk.
The volga risk is related to the $\nu$ parameter in the SABR model. 
The volga risk expresses the risk to the smile shape and curvature,
which we also saw during the investigating of the SABR model. 
This is also consistent with the interpretation of the $\nu$ parameter. 
Exactly that $\nu$ is the volatility of the volatility parameter $\alpha$.
\newpage
\noindent
Finally we will look at the delta risk, we will denote
the delta risk $\Delta$. The delta risk is related to f, the 
forward rate or the price of the underlying asset. 
During the investigating of the SABR model, we made
some choice regrading the determining of the alpha parameter.
This choice also affects the delta risk. So we note that
we chose the approach of determining the alpha, from solving the
$\sigma_{\text{ATM}}$, where $\sigma_{\text{ATM}}$ is given as 
listed in \autoref{sigma_ff_risk} above. 
Then we differentiate with respect to f, to obtain the delta risk. 
\begin{align}
    \Delta = \frac{\partial \text{BS}}{\partial f} +
     \frac{\partial \text{BS}}{\partial \sigma_\beta} 
     \Big({\frac{\partial \sigma_\beta(K; f, \alpha, \beta, \rho, \nu)}
     {\partial f}  + \frac{\partial \sigma_\beta(K; f, \alpha, \beta,
     \rho, \nu)}{\partial \alpha} 
     \frac{\partial \sigma(\text{ATM}, f)}{\partial f}}\Big)
\end{align}
The first term represents the ordinary delta risk calculated from 
the Black's model. The second term adds the SABR model's
 specific adjustment for the change in implied volatility, 
$\sigma_B$, caused by changes in the forward price f.
The last term represents the adjustment needed to maintain a constant 
$\sigma_{\text{ATM}}$ while f changes.
We also note that the last term will be zero, if $\beta=1$ \cite{Smile}.
\\\\
Then we would like to summarize one of the risks related to changes in the parameters of the SABR model.
Changes in the $\alpha$ parameter affect the level of the implied volatility smile. 
Higher $\alpha$ values increase the overall level of implied volatility, while lower values decrease it.
As previously discussed, it is common to fix $\beta$ since similar movements occur when changing both 
the $\beta$ and $\nu$ parameters. If we consider the effect of $\beta$ on the implied volatility smile, 
lower $\beta$ values make the smile steeper (more skewed), whereas higher $\beta$ values flatten the smile.
Additionally, higher $\nu$ values increase the curvature of the smile, making it more pronounced,
and vice versa. Positive $\rho$ values skew the smile to the right (towards higher strikes), 
as observed with the 10Y10Y EUR swaption. Conversely, negative $\rho$ values skew the smile to the 
left (towards lower strikes), as seen in our SABR model parameter analysis (\autoref{fig:rho}).
Finally, the forward price, $f$, adjusts the center of the implied volatility smile. 
Awareness of these movements in implied volatility is crucial if swaptions are included in asset allocation.
\\\\
So now we have covered all the risk or sensitives related to the SABR model. 
The tendencies we saw during the investigating of the SABR model, reflects
the same patterns regrading the various risk related to the model. 