\section{ The SABR model}
The SABR (Stochastic Alpha, Beta, Rho) model marks a pivotal advancement 
in financial modeling, effectively addressing the significant limitations 
found in traditional methods like the Black Scholes model, which presupposes 
constant volatility. Created in 2002 by Patrick Hagan, Deep Kumar, 
Andrew Lesniewski, and Diana Woodward, the SABR model is highly esteemed for 
its adeptness at managing the dynamic and unpredictable nature of market 
volatility.
\\\\
As a two-factor model, the SABR framework models both the forward rate 
(or asset price) and its volatility as stochastic processes. This approach 
is vital as it incorporates a stochastic behavior in volatility, significantly 
improving the model's ability to capture the true, skewed, and heavy-tailed 
nature of financial market data. By allowing for volatility fluctuations, 
the SABR model provides a flexible and realistic framework for pricing 
derivatives, proving especially useful for options with long maturities where 
the assumption of constant volatility falls short \cite{Smile}.

\subsection{Process for the forward rate}
The main different between the SABR model and the 
Black Scholes model is the assumptions regrading the 
volatility, as mentioned earlier. In the Black Scholes 
model the volatility is a assumed to be constant and 
in the SABR model the volatility evolve as a function
of time, t, the strike price, K, and the current
forward price, $f_t$. Futhermore the volatility itself
is random. So we chose the unknown coefficient $C(t,*)$
to be $\hat{\alpha} \hat{F}^{\beta}$, where the 
"volatility" $\hat{\alpha}$ is a stochastic process itself. 
The extra randomness is scaled thought the inclusion 
of a "volatility of volatility" parameter $\nu$.
\\\\
Now we will formulate the SABR model mathematically. 
The SABR model consist of a dynamic for the forward price
and one for the volatility, since the SABR model is a 
two-factor model. The SABR model also formulate the 
how the to process is correlated \cite{Smile}. 


\subsection{SABR Implied Volatility and Option Prices}

\subsection{Estimating Parameters}