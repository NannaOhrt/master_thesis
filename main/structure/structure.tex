\documentclass{article}
\usepackage[utf8]{inputenc}
\usepackage[english]{babel}
\usepackage[T1]{fontenc}
\usepackage{amsmath,amssymb,amsfonts,amsthm}
\usepackage{ulem}
\usepackage{subcaption}
\usepackage{caption}
\usepackage{gauss}
\usepackage{mathtools}
\usepackage{mathrsfs}
\usepackage{hyperref}
\usepackage{bbm}
\usepackage{graphicx} %Inkl. billeder osv.
\usepackage{enumerate} %Lister
\usepackage{fancyhdr} %Enables sidehoved og -fod.
\usepackage{lastpage} %Bruges til at angive antal sider
\usepackage[margin=1in,footskip=0.25in]{geometry}
\usepackage{pdfpages}
\pagestyle{fancy}
%\fancyfoot[C] { } %Sletter sidetal i C
\setcounter{MaxMatrixCols}{20}
\linespread{1,2}
\newcommand{\pic}[3]{\makebox[\textwidth]{\textbf{#1}}
\\
\begin{center}
\includegraphics[scale=#2,center]{#3}
\end{center}}
\newcommand{\QQ}{\mathbb{Q}}
\newcommand{\RR}{\mathbb{R}}
\newcommand{\BB}{\mathbb{B}}
\newcommand{\NN}{\mathbb{N}}
\newcommand{\FF}{\mathbb{F}}
\newcommand{\ZZ}{\mathbb{Z}}
\newcommand{\CC}{\mathbb{C}}
\newcommand{\EE}{\mathbb{E}}
\newcommand{\indi}[1]{\mathbbm{1}_{#1}}
\newcommand{\limi}[2]{\liminf\limits_{#1\rightarrow#2}}
\newcommand{\bcup}[1]{\bigcup\limits_{#1}}
\newcommand{\bcap}[1]{\bigcap\limits_{#1}}
\newcommand{\m}{\cdot}
\newcommand{\TF}[2]{\{#1_{#2}\}_{#2\in\mathbb{N}}}
\newcommand{\Ma}[2]{\left( \begin{array}{{#1}}
#2
\end{array} \right)}
\newcommand{\Mp}[1]{\begin{pmatrix}
    #1
    \end{pmatrix}}
\newcommand{\M}[1]{\begin{matrix}
    #1
    \end{matrix}}
\newcommand{\ver}[2]{#1\hspace{0.05cm}\vert\hspace{0.03cm}#2}
\newcommand{\mgd}[2]{\left\{#1 \hspace{0.1cm}\left|\hspace{0.1cm}#2\right.\right\}}
\newcommand{\llim}[2]{\lim\limits_{#1\rightarrow #2}}
\newcommand{\ls}{\limsup\limits_{n\rightarrow\infty}}
\newcommand{\li}{\liminf\limits_{n\rightarrow\infty}}
\newcommand{\Part}[2]{\frac{\partial{#1}}{\partial{#2}}}
\newcommand{\Int}[4][x]{\int_{#2}^{#3} #4 \hspace{0.05cm}\mathrm{d} #1}
\newcommand{\myeq}[2][=]{\mathrel{\overset{\makebox[0pt]{\mbox{\normalfont\tiny\sffamily $#2$}}}{#1}}}
\newcommand{\Myeq}[2][\leq]{\mathrel{\overset{\makebox[0pt]{\mbox{\normalfont\tiny\sffamily $#2$}}}{#1}}} 
\newcommand{\mb}[1]{\mathbb{#1}}
\newcommand{\mc}[1]{\mathcal{#1}}
\newcommand{\ms}[1]{\mathscr{#1}}
\newcommand{\as}{\myeq[\rightarrow]{as}}
\newcommand{\As}{\myeq[\sim]{as}}
\newcommand{\wk}{\myeq[\rightarrow]{wk}}
\newcommand{\Pkonv}{\myeq[\rightarrow]{P}}
\newcommand{\Dkonv}{\myeq[\rightarrow]{D}}
\newcommand{\Lkonv}[1][1]{\myeq[\rightarrow]{\mc{L}^{#1}}}
\newcommand{\borel}{\ms{B}(\mb{R})}
\newcommand{\inte}[3]{\int_{#1} #2 \hspace{0.05cm}\mathrm{d}{#3}}
\newcommand{\Norm}[1]{\left\lVert#1\right\rVert}
\newcommand{\norm}[1]{\left\lvert#1\right\rvert}
\newcommand{\Sum}[2]{\sum\limits_{#1}^{#2}}
\title{Master Thesis}
\author{}
\makeatletter
\providecommand*{\cupdot}{%
  \mathbin{%
    \mathpalette\@cupdot{}%
  }%
}
\newcommand*{\@cupdot}[2]{%
  \ooalign{%
    $\m@th#1\bigcup$\cr
    \hidewidth$\m@th#1\cdot$\hidewidth
  }%
}
\makeatother
\begin{document}
\maketitle
\tableofcontents
\newpage

\section*{Title}
Swaptions pricing 

\section*{Thesis statement}
In this thesis, I will investigate asset allocation with respect to swaptions and the affect different swaptions strategies has.
This analysis receivers a model selection to price swaptions and the different strategies will be back tasted on data.

\section*{Structure}

\section{Asset allocation}
\begin{itemize}
  \item Write about different assets.
  \item drawdown plot on some equites, bonds and implied volatility
  \item inflation plot on the same equites, bonds and implied volatility
\end{itemize}  
The idea is to give a motivation for swaptions as derivative in asset allocation. 
 

\section{Swaptions}
Introduction to swaptions

\section{Pricing swaptions}
Setup the framework for what is need to price swaptions.
Theory on pricig a swaption 

\section{risk neutral pricing}
Introduce risk neutral pricing, so in the end it is possible to price swaptions

\section{Black model}
Theory of the Black model. 
In the Black model the sigma is constant. 
Comment on there is a sigma - volatility, and what should this sigma be. Transition to introducing the SABR model.

\section{SABR model - implied volatility}
Theory of the SABR model 

Given market data on ATM volatility such as 10Y10Y ATM NORMAL EUR, we will calibrate the parameters in the SABR model. 
Then the calibrated parameters can be used to find the "sigma" the implied volatility, 
This "sigma" can be used in the Black model to price swaptions. 

\section{Risk premium}
Introduce how risk premium are calculated, we can perform two different swaptions strategies 
\\\\
Risk premium = expected return - risk-free rate 

\section{Two strategies}

10Y 10Y ATM EUR swaption - 20 years data and 3M 3M ATM EUR swaption - 20 years data 
\medskip
Maybe also for USD swaptions
\medskip
The goals is to find that swaption make good sense when you have a strategy when the swaptions has a long duration. 

\section{Data}
volatility  - given in BPS/DAY
\begin{itemize}
  \item 10Y 10Y EUR normal vol 
  \item 20Y 10Y EUR normal vol 
  \item 3M 3M EUR normal vol
  \item 1Y 1Y EUR normal vol 
  \item 10Y 10Y USD normal vol 
  \item 20Y 10Y USD normal vol 
  \item 3M 3M USD normal vol
  \item 1Y 1Y USD normal vol 
\end{itemize}
risk-free rates - given i BPS/ANNUM
\begin{itemize}
  \item 10Y 10Y EUR normal annual RFR vol 
  \item 20Y 10Y EUR normal annual RFR vol  
  \item 3M 3M EUR normal annual RFR vol 
  \item 1Y 1Y EUR normal annual RFR vol  
  \item 10Y 10Y USD normal annual RFR vol  
  \item 20Y 10Y USD normal annual RFR vol  
  \item 3M 3M USD normal annual RFR vol 
  \item 1Y 1Y USD normal annual RFR vol  
\end{itemize}


\section{}

Swaption as now asset

Trading as a active stategies, short vs long term. 

SSA, is there a risk premium

onlu exspure is the vol, same exspure in the hold time

look at the distribution on the of the time series. 

positive middel value. 

Can the parameters in the data be contruct as the same as the estimate. 


IDE! 
Write about different model, black model and SABR model. 
The goal is to find out is there is a risk premium in general!
estimate the parameters in the model. 
- over time, in the vol OTM and the ATM. 
how to construct strategies
-how to get the risk premium.


\section{Plan}

Write down the idea of this thesis \\
Make a outline for this thesis \\
Make a list of the data we need \\
Get all the data \\
Start writing theory 

\end{document}